\documentclass[12pt,%
    %anonym,
    % times,
    %referee,
    % doublespacing,   
]{lin-v2/lin}
\usepackage[utf8]{inputenc}
\usepackage[T1]{fontenc}

\usepackage{amssymb}
\usepackage{tipa}

\usepackage{listings}

\usepackage[noglossaries]{leipzig}
\usepackage{gb4e}
\noautomath

\usepackage{rrgtrees}
\usepackage{pst-node}
\usepackage{pst-tree}

\usepackage{booktabs}
\usepackage{hyperref}

\usepackage[section]{placeins}

\let\Oldsubsection\subsection
\renewcommand{\subsection}{\FloatBarrier\Oldsubsection}

\let\Oldsubsubsection\subsubsection
\renewcommand{\subsubsection}{\FloatBarrier\Oldsubsubsection}


\newcommand{\pref}[1]{(\ref{#1})}
\newleipzig{mid}{mid}{middle}
\newleipzig{count}{count}{count\-able noun}
\newleipzig{mass}{mass}{mass noun}
\newleipzig{fin}{fin}{finite verb}
\newleipzig{pro}{pro}{pro\-noun}
\newleipzig{prep}{prep}{pre\-position}


\usepackage{color}
\definecolor{mygreen}{rgb}{0,0.6,0}
\definecolor{mygray}{rgb}{0.5,0.5,0.5}
\definecolor{mymauve}{rgb}{0.58,0,0.82}

\lstset{
    backgroundcolor=\color{white},
    basicstyle=\footnotesize\ttfamily,
    belowskip=2em,
    breaklines=true,
    captionpos=t,
    commentstyle=\color{mygreen},
    extendedchars=true,
    frame=single,
    keepspaces=true,
    keywordstyle=\color{blue},
    language=Python,
    literate=
  {á}{{\'a}}1 {é}{{\'e}}1 {í}{{\'i}}1 {ó}{{\'o}}1 {ú}{{\'u}}1
  {Á}{{\'A}}1 {É}{{\'E}}1 {Í}{{\'I}}1 {Ó}{{\'O}}1 {Ú}{{\'U}}1
  {à}{{\`a}}1 {è}{{\`e}}1 {ì}{{\`i}}1 {ò}{{\`o}}1 {ù}{{\`u}}1
  {À}{{\`A}}1 {È}{{\'E}}1 {Ì}{{\`I}}1 {Ò}{{\`O}}1 {Ù}{{\`U}}1
  {ä}{{\"a}}1 {ë}{{\"e}}1 {ï}{{\"i}}1 {ö}{{\"o}}1 {ü}{{\"u}}1
  {Ä}{{\"A}}1 {Ë}{{\"E}}1 {Ï}{{\"I}}1 {Ö}{{\"O}}1 {Ü}{{\"U}}1
  {â}{{\^a}}1 {ê}{{\^e}}1 {î}{{\^i}}1 {ô}{{\^o}}1 {û}{{\^u}}1
  {Â}{{\^A}}1 {Ê}{{\^E}}1 {Î}{{\^I}}1 {Ô}{{\^O}}1 {Û}{{\^U}}1
  {œ}{{\oe}}1 {Œ}{{\OE}}1 {æ}{{\ae}}1 {Æ}{{\AE}}1 {ß}{{\ss}}1
  {ű}{{\H{u}}}1 {Ű}{{\H{U}}}1 {ő}{{\H{o}}}1 {Ő}{{\H{O}}}1
  {ç}{{\c c}}1 {Ç}{{\c C}}1 {ø}{{\o}}1 {å}{{\r a}}1 {Å}{{\r A}}1
  {€}{{\euro}}1 {£}{{\pounds}}1 {«}{{\guillemotleft}}1
  {»}{{\guillemotright}}1 {ñ}{{\~n}}1 {Ñ}{{\~N}}1 {¿}{{?`}}1
  {ð}{{\dh}}1 {Ð}{{\Dh}}1,
  numbers=left,
  numbersep=5pt,
  numberstyle=\tiny\color{mygray},
  rulecolor=\color{black},
  stringstyle=\color{mymauve}
}


\begin{document}
    
\leftrunning{}  % Short author list

\rightrunning{} % Short title

\title{An RRG Parser for Middle Constructions in Faroese}

\author[1]{\givenname{Sigmund} \surname{Vestergaard} -- B00108126}

\address[1]
{%
  \inst{Institute of Technology Blanchardstown}, % Institution name should be in \inst
  \addr{Blanchardstown Road North}, % Street 
  \addr{Dublin 15}, % Postcode etc
  \cnty{Ireland}  % Country
  \email{B00108126@student.itb.ie} % email
}

\maketitle

\begin{abstract}
    In this article we describe a parser for Middle Constructions
    in Faroese built on Role and Reference Grammar (RRG).
    \keywords{RRG, Faroese, Middle Constructions, Parsing}
\end{abstract}


\section{Introduction}

In this paper we describe the architecture and implementation of a Role and Reference Grammar (RRG)
parser for middle constructions in Faroese. The parser will be able to take a so-called "middle"
sentence and tell us whether it is grammatical or not, and if it is, show an RRG tree for it.

In the first section below we will briefly describe the morphology
and syntax of Faroese in general. The description is based on~\citep{faroese}.
We will then move on to describe middle constructions in Faroese in the second section.
In Section 2 we will also analyse middle constructions in terms of
Role and Reference Grammar (RRG). After this we will 
describe the design of the lexicon in Section 3 and the parser itself in Section 4.
We will finish off in Sections 5 and 6 by testing the parser.


\section{Faroese Morphology and Syntax}

Faroese is derived from the Norse language of the primarily Norwegian settlers
who moved to the islands in the ninth century. Due to a lack of Viking Age
and medieval sources little is known about the development of Faroese into
the language we know today. Nothing of substance was written in Faroese
until the 1770s, by which time most the features of the modern language
must have developed.

\subsection{Nominal categories}

Like Icelandic, Faroese has three grammatical genders; masculine, feminine, and neuter.
Faroese nouns inflect for gender, but also number (singular/plural) and case (nominative/accusative/dative/genitive).
The inflectional endings vary according to gender. Definiteness of nouns is indicated by a suffixed article.

Just like nouns, adjectives also inflect for number, case, and gender. Additionally they also inflect for degree
(positive/comparative/superlative). Adjectives typically have two forms of inflections - strong or weak - depending
on the definiteness of the noun phrase they are a part of.

Finally, articles, pronouns, the cardinal numbers 1 to 3, and the ordinal numbers, inflect for number, case, and gender.

Faroese noun phrases show extensive agreement, including number agreement between nouns and the adjectives that modify them.
This holds for both attributive and predictive adjectives. We illustrate this using the noun \emph{drongur} 'boy'
with the adjective \emph{klókur} 'smart'~\citep[61]{faroese}:
\begin{exe}
    \ex
    \begin{xlist}
    \item \gll ein klókur drongur klókir drongir\\
    a smart boy.\Sg{} smart boy.\Pl{}\\
    \trans 'a smart boy' 'smart boys'
    \item \gll drongurin er klókur dreingirnir eru klókir\\
    boy-\Det{} is smart boy-\Det,\Pl{} is.\Pl{} smart.\Pl\\
    \trans 'the boy is smart' 'the boys are smart'
    \end{xlist}
\end{exe}

The gender of nouns is reflected in the different forms of the personal pronouns used to refer to them and the
gender of adjectives and articles used to modify said nouns. We illustrate this with the following example:
\begin{exe}
    \ex
    \gll Hetta er ein klókur drongur\\
    this is a.\M{} smart.\M{} boy.\M\\
    Hann. er klókur\\
    he.\M{} is smart.\M\\
    \gll Hetta er ein klók genta\\
    this is a.\F{} smart.\F{} girl.\F\\
    Hon er klók\\
    she.\F{} is smart.\F{}\\
    \gll Hetta er eitt klókt barn\\
    this is a.\N{} smart.\N{} child.\N\\
    Tað er klókt\\
    it.\N{} is smart.\N\\
\end{exe}

Old Norse and older Faroes had four morphologically distinctive cases - nominative, accusative, dative, and genitive -
but only the three first are productive in modern spoken Faroese (modern \emph{written} Faroese still retains the genitive
to a degree). We illustrate this with a small example:
\begin{exe}
    \ex
    \gll Gentan svav\\
    girl-\Det.\Nom{} sleep.\Pst\\
    \trans 'The girl slept'\\
    \gll Eg sá gentuna\\
    I see.\Pst{} girl-\Det.\Acc\\
    \trans 'I saw the girl'\\
    \gll Hetta er hundurin hjá gentuni\\
    This is dog-\Det{} with girl-\Det.\Dat\\
    \trans 'This is the girl'{}s dog'
\end{exe}

Despite this a genitive form can be produced for nouns and personal pronouns, but less so for adjectives. The genitive form
of personal pronouns is widely used while the genitive form of many nouns is found in fixed expressions and as the first
part of certain compounds, but it is uncertain whether speakers intuitively interpret these forms as genitive~\citep[62]{faroese}.

Instead of genitive, modern spoken Faroese prefers prepositional constructions involving a dative form of the noun as
illustrated in these examples:
\begin{exe}
    \ex
    \gll Her eru húsini hjá einum ríkum manni\\
    here is.\Pl{} house-\Det.\Pl{} with a.\Dat{} rich.\Dat{} man.\Dat{}\\
    'Here is a rich man'{}s house/home'
    \gll Kettlingurin hjá kettuni hjá mær er vakur\\
    kitten-\Det{} with cat-\Det{}.\Dat{} with I.\Dat{} is beautiful\\
    'My cat'{}s kitten is beautiful'
\end{exe}

Above we saw examples with the preposition \emph{hjá} 'with', but Faroese speakers also use other prepositions with the dative,
depending on the semantic function. We'll look at some examples~\citep{faroese}:
\begin{exe}
    \ex
    \gll takið á húsinum motorurin í bilinum\\
    roof-\Det{}.\Nom.\N{} on house-\Det{}.\Dat{}.\N{} motor-\Det{}.\Nom{}.\M{} in car-\Det.\Dat.\M\\
    \trans 'the roof of the house' 'the car'{}s engine'
    \gll abbi at dreinginum aldurin á kirkjuni\\
    grandfather to boy-\Det.\M.\Dat{} age-\Det on church-\Det.\Dat\\
    \trans 'the boy'{}s grandfather' 'the age of the church' 
    \gll halin á kúnni tenninar í hundinum\\
    tail-\Det on cow-\Det.\Dat{} tooth-\Det.\Pl{} in dog-\Det.\Dat\\
    \trans 'the cow'{}s tail' 'the dog'{}s teeth'
    \gll høvdið á mær eyguni í honum\\
    head-\Det on I.\Dat{} eyes-\Det{} in he.\Dat\\
    \trans 'my head' 'his eyes'
\end{exe}

With nouns denoting family relationships an accusative form is normally used instead of genitive or a prepositional phrase
as illustrated here:
\begin{exe}
    \ex
    \gll pápi dreingin mamma gentuna beiggi Jógvan\\
    father boy-\Det.\Acc{} mum girl-\Det.\Acc{} brother Jógvan.\Acc\\
    \trans 'the boy'{}s father' 'the girl'{}s mother' 'Jógvan'{}s brother'
\end{exe}

As mentioned above, adjectives can be grouped into two categories: \emph{strong} adjectives and \emph{weak} adjectives.
Which category they belong to depends on the definiteness of the noun phrase they form a part of. The general rule is that
the adjective takes the weak form if the noun phrase is definite, and the strong form if the noun phrase is indefinite.
We illustrate this with two examples~\citep[65]{faroese}:
\begin{exe}
    \ex
    \begin{xlist}
        \item \gll Hetta er ein stórur bilur og ein lítil bók.\\
        this is a big.\Nom.\Sg.\M{} car.\Nom.\Sg.\M{} and a small.\Nom.\Sg.\F{} book.\Nom.\Sg.\F\\
        \trans 'This is a big car and a small book'
        \item \gll Hetta er tann stóri bilurin og tann lítla bókin.\\
        this is the big.\Nom.\Sg.\M{} car-\Det.\Nom.\Sg.\M{} and the small-\Det.\Nom.\Sg.\F{} book\\
        \trans 'This is the big car and the small book'
    \end{xlist}
\end{exe}

Finally, most adjectives can be inflected for degree by adding the suffixes \emph{-(a)r} and \emph{-(a)st} in
comparative and superlative, respectively. Indeclinable adjectives express difference in degree by using the
auxiliary verbs \emph{meiri} 'more' (comparative) and \emph{mest} 'most' (superlative). We'll finish by showing a
couple of examples of this:
\begin{exe}
    \ex
    \begin{xlist}
        \item \gll gulur gul-a-ri gul-ast-ur\\
        yellow yellower yellowest\\
        \item \gll hóskandi meiri hóskandi mest hóskandi\\
        appropriate more appropriate most appropriate\\
    \end{xlist}
\end{exe}

\subsection{Verbal categories}

Faroese verbs are inflected by person, number, and tense, with the two following characteristics with respec to
person inflection~\citep[67]{faroese}:
\begin{enumerate}
    \item Faroese verbs do not show any person distinctions in the plural and regular (weak).
    \item Faroese verbs do not show any person distinctions neither in the singular nor in the past tense.
\end{enumerate}

This can be illustrated with the following example:
\begin{exe}
    \ex
    \begin{xlist}
        \item \gll eg kalli tú kallar hann/hon/tað kallar\\
        I call.\First\Sg.\Prs{} you call.\Second\Sg.\Prs{} he/she/it call.\Third\Sg.\Prs\\
        \trans 'I call' 'you call' 'he/she/it calls'
        \item \gll vit kalla tit kalla teir/tær/tey kalla\\
        we call.\First\Pl.\Prs{} you.\Pl{} call.\Second\Pl.\Prs{} they.\M/\F/\N{} call.\Third\Pl.\Prs\\
        \trans 'we call' 'you call' 'they call'
        \item \gll eg kallaði tú kallaði hann/hon/tað kallaði\\
        I call.\First\Sg.\Pst{} you call.\Second\Sg.\Pst{} he/she/it call.\Third\Sg.\Pst\\
        \trans 'I called' 'you called' 'he/she/it called'
    \end{xlist}
\end{exe}

Faroese has two distinct imperative forms, plural and singular, as illustrated here:
\begin{exe}
    \ex
    \gll Gev/gevið hesum manninum gætur!\\
    give.\Sg/\Pl{} this man-\Det.\Third\Sg\Dat{} attention\\
    \trans 'Give attention to this man!'
\end{exe}

The singular \emph{gev} would be used if addressing one person, and the plural \emph{gevið} if more than one person
is being addressed. There was no distinctive plural imperative in Old Norse, where the \Second\Pl{} indicative had this role,
but in Faroese there is a distinction between the default finite forms and the imperative forms. We illustrate this below,
where the non-imperative forms are referred to as \emph{indicative}, although it is uncertain that one can speak of indicative
in Faroese, because there is no productive contrasting subjunctive in Faroese~\citep[67-68]{faroese}:
\begin{exe}
    \ex
    \begin{xlist}
        \item \gll Tú fert til hús.\\
        you.\Second\Sg{} go.\Second\Sg.\Prs.\Ind{} to house\\
        \trans 'You go home'
        \item \gll Far til hús!\\
        go.\Second\Sg.\Imp{} to house\\
        \trans 'Go home!'
        \item \gll Tit fara til hús.\\
        you.\Second\Pl{} go.\Second\Pl.\Prs.\Ind{} to house\\
        \trans 'You go home'
        \item \gll Farið til hús!\\
        go.\Second\Pl.\Imp{} to house\\
        \trans 'Go home!'
    \end{xlist}
\end{exe}

We mentioned above that the subjunctive is not productive in Faroese anymore, and it should be added that only
a few relic forms exist in main clauses in relatively fixed expressions and in religious language. And where they
exist they almost exclusively express \emph{optative modality}, as illustrated below~\citep[68]{faroese}:
\begin{exe}
    \ex
    \begin{xlist}
        \item \gll Jesus fylgir tær\\
        Jesus follow.\Third\Sg.\Ind{} you\\
        \trans 'Jesus is with you'
        \item \gll Jesus fylgi tær\\
        Jesus follow.\Third\Sg.\Subj{} you\\
        \trans 'Jesus be with you' 
        \item \gll Gud signar Føroyar\\
        God bless.\Third\Pl.\Ind{} Faroes\\
        \trans 'God blesses the Faroes'
        \item \gll Gud signi Føroyar\\
        God bless.\Third\Pl.\Subj{} Faroes\\
        \trans 'God bless the Faroes'
    \end{xlist}
\end{exe}

Examples of other, relatively fixed optative forms, which aren't religious expressions, are:
\begin{exe}
    \ex
    \begin{xlist}
        \item \gll Hann leingi livi!\\
        He long live.\Third\Sg.\Subj{}\\
        \trans 'Long live he!'
        \item \gll Gævi at tað skjótt varð heystfrí!\\
        Give.\Third\Sg.\Pst.\Subj{} that it soon become.\Sg.\Pst{} {autumn break}\\
        \trans 'I wish we had autumn break soon!'
        \item \gll Hevði tað nú bara gingist henni væl.\\
        Have.\Third\Sg.\Pst.\Subj{} it now just go.\Sg.\Pst{} her well\\
        \trans 'I wish things would go well for her'
    \end{xlist}
\end{exe}

Of these the first one uses the present subjunctive while the two others use the past subjunctive (or what was the
past subjunctive in older Faroese).

Typically the passive is formed with the auxiliary verbs \emph{verða} 'be, become' and \emph{blíva} 'be, become'. 
The participle agree in case, gender, and number with a nominative subject, and the agent is more frequently left
out than in English. If the agent is included, it is with the auxiliary \emph{av} 'by', which takes a dative form.
We illustrate this with some examples~\citep[69]{faroese}:
\begin{exe}
    \ex
    \begin{xlist}
        \item \gll Hann kysti hana\\
        he.\Third\Sg.\Nom{} kiss.\Sg.\Pst{} her.\Third\Sg\Acc\\
        \trans 'He kissed her'
        \item \gll Hon varð/bleiv kyst (av honum)\\
        she.\Third\Sg.\Nom{} be.\Sg.\Pst.\Aux{} kiss.\Nom.\Sg.\F.\Pst.\Ptcp{} (by him.\Dat)\\
        \trans 'She was kissed by him'
    \end{xlist}
    \ex
    \begin{xlist}
        \item \gll Hon kysti teir\\
        she.\Third\Sg\Nom{} kiss.\Sg.\Pst.\Ind{} them.\Acc.\Pl.\M{}\\
        \trans 'She kissed them'
        \item \gll Teir vórðu/blivu kystir (av henni)\\
        they.\Third\Pl.\Nom.\M{} be.\Pl.\Pst.\Aux{} kiss.\Nom.\Pl.\M{} (by her.\Dat)\\
        \trans 'They were kissed by her'
    \end{xlist}
    \ex
    \begin{xlist}
        \item \gll Teir smurdu hann av\\
        they smear he.\Third\Sg.\Acc{} off\\
        \trans 'They beat him up'
        \item \gll Hann varð/bleiv avsmurdur\\
        he.\Third\Sg.\Nom{} be offsmear.\Nom.\Sg.\M.\Pst.\Ptcp\\
        \trans 'He was beaten up'
    \end{xlist}
\end{exe}

In Faroese it is frequently possible to form so-called \emph{-st}-forms, or \emph{middle} forms, by adding the suffix
\emph{-st} to various inflectional forms of the verb. The meaning of the middle forms varies widely in Faroese, but
the usages most frequently mentioned in discussions of the middle forms are reflexive, reciprocal, or passive.
Here we give some examples, but we will discuss the middle forms in more detail in the next section:
\begin{exe}
    \ex
    \begin{xlist}
        \item \gll Eg settist niður\\
        I sit down\\
        \trans 'I sat down' (reflexive meaning)
        \item \gll Teir berjast altíð\\
        they fight always\\
        \trans 'They always fight' (reciprocal meaning)
        \item \gll Oyggin kallast Nólsoy\\
        island-\Det{} call Nólsoy\\
        \trans 'The island is called Nólsoy' (passive meaning)
    \end{xlist}
\end{exe}

Perfect tense is either formed with the auxiliary \emph{hava} 'have' and the supine (\Sg\N{} of \Pst\Ptcp{}) of the main verb,
or it is formed with the auxiliary \emph{vera} 'be' and the inflected and agreeing past participle. \emph{Hava} is used
with all transitive verbs and most intransitive verbs~\citep[72]{faroese}:
\begin{exe}
    \ex
    \begin{xlist}
        \item \gll Hon hevur lisið bókina.\\
        she have.\Third\Sg.\Pst{} read.\textsc{sup} book-\Det.\Acc\\
        \trans 'She has read the book'
        \item \gll Teir hava sovið leingi.\\
        they have.\Third\Pl.\Pst{} sleep.\textsc{sup} long\\
        \trans 'They have slept for long'
        \item \gll Hann hevur verið ríkur.\\
        he have\Third\Sg.\Pst{} be.\textsc{sup} rich\\
        \trans 'He has been rich'
        \item \gll Hann er vorðin ríkur.\\
        he.\Nom.\Sg.\M{} is become.\Nom.\Sg.\M{} rich\\
        \trans 'He has become rich'
    \end{xlist}
\end{exe}

Past perfect is formed with past tense of the relevant auxiliary (\emph{vera/hava}as mentioned above), and the perfect
passive is formed with the auxiliary \emph{vera} 'be', not \emph{hava} 'have'. We illustrate this with a couple of examples:
\begin{exe}
    \ex
    \begin{xlist}
        \item \gll Hann hevði verið ríkur.\\
        he have.\Pst{} be.\textsc{sup} rich.\Nom.\M{}\\
        \trans 'He had been rich'
        \item \gll Hann var vorðin ríkur.\\
        he be.\Third\Sg.\Pst{} become.\Pst.\Ptcp{} rich.\Nom.\M{}\\
        \trans 'He had become rich'
    \end{xlist}
    \ex
    \gll Hann er/*hevur ofta vorðin/blivin avsmurdur.\\
    he be.\Third\Sg.\Prs{}/*have often become.\Pst.\Ptcp{} off-smear.\Nom.\M{}\\
    \trans 'He has often been beaten up'
\end{exe}

The indicative-subjunctive distinction is not productive in Faroese and past subjunctive forms generally do not exist.
Past subjunctive was commonly used in Old Norse (and still is in Modern Icelandic) to indicate a counterfactual or hypthetical
situation. The regular past tense can have this function is Faroese, but the meaning of such forms are typically
ambiguous. We illustrate this with a few examples:
\begin{exe}
    \ex
    \begin{xlist}
        \item \gll Eg gjørdi tað fegin.\\
        I do.\Pst{} it gladly\\
        \trans 'I did it gladly.' or 'I would gladly do it.'
        \item \gll Hann hevði dripið hundin.\\
        he have.\Pst{} kill.\Pst.\Ptcp{} dog-\Det.\Acc.\M{}\\
        \trans 'He had killed the dog.' or 'He would have killed the dog.'
        \item \gll Hann tók bókina.\\
        he take.\Pst{} book-\Det.\Nom.\F{}\\
        \trans 'He took the book.' or 'He would gladly take the book if...'
        \item \gll Eg hevði fegin gjørt tað, um eg fekk pengar fyri tað.\\
        I have.\Pst{} gladly do.\Pst.\Ptcp{} it if I get.\Pst{} money for it\\
        \trans 'I would gladly have done it if I was paid for it.'
        \item \gll Hann drap hundin, um hann fekk hendur á honum.\\
        he kill.\Pst{} dog-\Det{} if he get.\Pst{} hand.\Nom.\Pl. on it\\
        \trans 'He would kill the dog if he got his hands on it.'
        \item \gll Hann hevði tikið bókina frá mær, um hann hevði sæð meg lisið í henni.\\
        he have.\Pst{} take.\Pst.\Ptcp{} book-\Det{}.\Nom{} from me if he have.\Pst{} see.\Pst.\Ptcp{} me read.\Pst.\Ptcp{} in it\\
        \trans 'He would have taken the book from me if he had seen me reading it.'
    \end{xlist}
\end{exe}

The default word order in Faroese is subject-verb-object (SVO) or subject-auxiliary-main verb-object (SAVO), both in main clauses
and embedded ones. We look at some examples~\citep[236]{faroese}:
\begin{exe}
    \ex
    \begin{xlist}
        \item \gll Jógvan las bókina.\\
        Jógvan read.\Pst{} book-\Det.\Acc.\Sg.\F{}\\
        \trans 'Jógvan read the book.'
        \item \gll Jógvan hevur lisið bókina.\\
        Jógvan have.\Pst{} read.\Pst.\Ptcp{} book-\Det.\Acc.\Sg.\F{}\\
        \trans 'Jógvan has read the book.'
        \item \gll Eg haldi, at Jógvan hevur lisið bókina.\\
        I think.\Prs{} that Jógvan have.\Pst{} read.\Pst.\Ptcp{} book-\Det\\
        \trans 'I think that Jógvan has read the book.'
    \end{xlist}
\end{exe}

As a rule, the indirect object precedes the direct object and typically appears in the dative, although
indirect objects in the accusative form also appear. Lets look at some examples:
\begin{exe}
    \ex
    \begin{xlist}
        \item \gll Turið gav Hjalmari nógvar bøkur.\\
        Turið.\Nom{} give.\Pst{} Hjalmar.\Dat{} many book.\Nom.\Pl\\
        \trans 'Turið gave Hjalmar many books.'
        \item \gll Eg spurdi, um Zakaris seldi Eivindi tann gamla bilin.\\
        I ask if Zakaris.\Nom{} sell.\Pst{} Eivind.\Dat{} the old car-\Det.\Acc\\
        \trans 'I asked if Zakaris sold the old car to Eivind.'
        \item \gll Hon lærdi meg niðurlagið.\\
        she teach.\Pst{} me.\Acc refrain-\Det.\Acc\\
        \trans 'She taught me the refrain.'
    \end{xlist}
\end{exe}

Should we move an object or a prepositional phrase, or some other non-subject, to the front of a sentence, as is done
in Topicalisation, the
finite verb shows up in second place followed by the subject, i.e. Faroese is a "verb-second" (V2) language like the
other Germanic languages except English. We take a look at some examples~\citep[238-239]{faroese}:
\begin{exe}
    \ex
    \begin{xlist}
        \item \gll \emph{Hesa bókina} hevur Jógvan lisið.\\
        {this book}-\Det.\Acc{} have.\Pst{} Jógvan.\Nom{} read.\Pst.\Ptcp{}\\
        \trans 'This book has Jógvan read.'
        \item \gll \emph{Tann gamla bilin} seldi Zakaris Eivindi.\\
        {the old car}-\Det.\Acc{} sell.\Third.\Sg.\Pst{} Zakaris.\Nom{} Eivind.\Dat.\\
        \trans 'The old car Zakaris sold to Eivind.'
        \item \gll \emph{Jóannes} haldi eg eigur hesa bókina.\\
        Jóannes.\Nom{} think I own.\Third\Sg.\Prs{} this book-\Det.\Acc\\
        \trans 'Jóannes, I think, owns this book.'
    \end{xlist}
\end{exe}

As is typical for modern Germanic languages, adjectives precede the noun they modify:
\begin{exe}
    \ex
    \begin{xlist}
        \item \gll ein \emph{vøkur} genta, ein \emph{bláur} bilur\\
        a beautiful.\Nom.\F{} girl.\Nom.\F{}, a blue\Nom.\M{} car.\Nom.\M{}\\
        \trans 'a beautiful girl', 'a blue car'
        \item \gll Hann kom súkklandi á einari \emph{gamlari} súkklu.\\
        he come.\Pst{} ride.\Prs.\Ptcp{} on an old.\Dat.\F{} bicycle.\Dat.\F{}\\
        \trans 'He came riding on an old bicycle.'
    \end{xlist}
\end{exe}

When it comes to adverbs we can broadly distinguish between three basic adverbial positions~\citep[241]{faroese}:
\begin{enumerate}
    \item the medial position, following the finite verb
    \item the verb phrase position, following a possible object and other elements of the verb phrase
    \item the modifying position, when the adverb is modifying and adjective and other adverbs
\end{enumerate}

We illustrate this with a few examples:
\begin{exe}
    \ex
    \begin{xlist}
        \item \gll Tey hava \emph{ikki/ivaleyst/jú/aldri} lisið bókina.\\
        they have not/undoubtedly/actually/never.\Adv{} read book-\Det.\Acc\\
        \trans 'They have not/undoubtedly/actually/never read the book.'
        \item \gll Tey hava lisið bókina \emph{tá/har/væl og virðiliga}.\\
        they have read book-\Det.\Acc{} {then/there/well and thoroughly}.\Adv{}\\
        \trans 'They have read the book then/there/well and thoroughly.'
        \item \gll Tey hava lisið hesa ógvuliga longu bókina sera væl.\\
        they have read this extremely.\Adv{} long book-\Det.\Acc{} very.\Adv{} well\\
        \trans 'They have read this extremely long book very well.'
    \end{xlist}
\end{exe}

Let's end by looking at the case of subject, object, and indirect object. The regular \emph{subject} case in Faroese
is \emph{nominative}, as we already have seen in previous examples:
\begin{exe}
    \ex
    \begin{xlist}
        \item \gll \emph{Hann} skrivar.\\
        he.\Nom.\Sg{} write.\Third\Sg\\
        \trans 'He writes.'
        \item \gll \emph{Hon} arbeiðir.\\
        she.\Nom.\Sg{} work.\Third\Sg\\
        \trans 'She works.'
        \item \gll \emph{Børnini} spæla.\\
        child-\Det.\Nom.\Pl play.\Third\Pl\\
        \trans 'The children play'
        \item \gll \emph{Vit} settu niður epli {í gjár}.\\
        we.\Nom.\Pl{} put down potato.\Pl{} yesterday\\
        \trans 'We planted potatoes yesterday.'
    \end{xlist}
\end{exe}

Before we continue, it should be mentioned that some verbs take non-nominative subjects in modern Faroese,
but we won't look further into that here.

Looking at \emph{direct object} case we can say that the \emph{accusative} is the default case,
as we have already seen in previous examples:
\begin{exe}
    \ex
    \begin{xlist}
        \item \gll Hon keypti bókina.\\
        she buy.\Pst{} book-\Det.\Acc\\
        \trans 'She bought the book.'
        \item \gll Hann seldi telduna.\\
        he sell.\Pst{} computer-\Det.\Acc\\
        \trans 'He sold the computer.'
    \end{xlist}
\end{exe}

It should be mentioned, however, that a number of verbs take a direct object in dative case, which we also touched upon earlier,
but we will restrict ourselves here to saying that among the verbs that take a direct object in the dative are
verbs of helping, ordering, praising, thanking, welcoming, etc., leading one to associate the dative objects with
thematic roles such as recipients and experiencers. The semantic verb classes and thematic roles we have mentioned are roughly
the same as the ones taking a dative object in Icelandic, although dative objects are getting more rare in modern Faroese
while still being prevalent in Icelandic~\citep[257-258]{faroese}.

At last we have the \emph{indirect object}, which takes the \emph{dative} as its default case, as we have seen in previous examples.
Among the verbs taking an indirect object are verbs meaning 'sell', 'lend', 'give', 'send', etc. Let's look at three examples:
\begin{exe}
    \ex
    \begin{xlist}
        \item \gll Hann beyð henni starv.\\
        he offer.\Pst{} she.\Dat{} job.\Acc\\
        \trans 'He offered her a job.'
        \item \gll Tey fingu sær bil.\\
        they get.\Pst{} themselves.\Dat{} car.\Acc\\
        \trans 'They got themselves a car.'
        \item \gll Fyrigev honum syndir hansara.\\
        forgive.\Imp{} him.\Dat{} sin.\Pl.\Acc{} he.\Gen\\
        \trans 'Forgive him his sins.'
    \end{xlist}
\end{exe}

We notice from the examples that these verbs, \emph{bjóða} 'offer', \emph{fáa} 'get', \emph{fyrigeva} 'forgive', are
examples of verbs that take two objects, one dative (indirect) object and one accusative (direct) object. We call these verbs \emph{ditransitive}.
This dative--accusative pattern is also the most common case marking pattern of ditransitive verbs in Icelandic, but there
we also have many other patterns, such as dative--dative, dative--genitive, accusative--dative, accusative--genitive, and
accusative--accusative, most of which are not found in Faroese~\citep[262-263]{faroese}.


\section{Middle Constructions in Faroese}

Above we briefly touched on middle forms in Faroese, but since they are the main subject of this paper we need to look closer at them.

We mentioned that the middle constructions are formed by adding \emph{-st} to various inflectional forms of the verb,
but let's now look further at the origin and semantics of the middle constructions.

The origin of the \emph{-st} suffix is considered to be the reflexive pronoun \emph{sik} in Old Norse,
which corresponds to the reflexive \emph{seg} in modern Faroese. Thus Old Norse \emph{setja sik} became
\emph{setjast} 'sit down', possibly via the intermediate form \emph{setjask}.
In modern Faroese this evolved to \emph{setast} (the Old Norse verb \emph{setja} became \emph{seta} in
modern Faroese)~\citep[277]{faroese}.

As we mentioned in the previous section, the \emph{-st}-forms can have a wide variety of meanings,
i.e. the semantics vary according to what verb is used, and we'll now look at the various meanings
while giving some glossed examples (we add \emph{+st} to the gloss to 
indicate an \emph{-st}-form)~\citep[277-278]{faroese}.

\subsection*{Reflexive meaning}

\begin{exe}
    \ex
    \begin{xlist}
        \item \gll Eg settist niður.\\
        I sit+st.\Pst{} down\\
        \trans 'I sat down.'\\
        \item \gll Tey vandust skjótt við hitan.\\
        They {get-used}+st.\Pst{} soon with heat-\Def{}\\
        \trans 'They soon got used to the heat.'
    \end{xlist}
\end{exe}

\subsection*{Reciprocal meaning}

\begin{exe}
    \ex
    \begin{xlist}
        \item \gll Maðurin og konan heilsaðust.\\
        Man-\Def{} and woman-\Def{} greet+st.\Pst\\
        \trans 'The man and the woman greeted each other.'\\
        \item \gll Tey bítast og klórast.\\
        They bit+st.\Prs{} and scratch+st.\Prs\\
        \trans 'They bite and scratch each other.' (i.e. 'They fight.')
    \end{xlist}
\end{exe}

\subsection*{Modal meaning}

\begin{exe}
    \ex
    \begin{xlist}
        \item \gll Ikki slepst uppaftur uttan hjálp.\\
        not get+st.\Prs{} {upp again} without help\\
        \trans 'One cannot get back up without help.'\\
        \item \gll Ikki kemst uttanum, at málið hevur týdning.\\
        not get+st.\Prs{} around that language-\Def{} have.\Prs{} importance\\
        \trans 'One cannot get around the fact that language is important.'\\
    \end{xlist}
\end{exe}

\subsection*{Middle (or passive) meaning}

\begin{exe}
    \ex\label{middles}
    \begin{xlist}
        \ex\label{middles1} \gll Íbúð ynskist til leigu.\\
        apartment wish+st.\Prs{} to rent\\
        \trans 'Apartment for rent is sought.'\\
        \ex\label{middles2} \gll Bókin seldist væl.\\
        book-\Def{} sell+st.\Pst{} well\\
        \trans 'The book sold well.'\\
        \ex\label{middles3} \gll Her skal eitt hús byggjast.\\
        here shall a house build+st.\Prs{}\\
        \trans 'A house is to be build here.'\\
        \ex\label{middles4} \gll Hon hoyrdist syngja langa leið.\\
        she {hear}+st.\Pst{} sing long way\\
        \trans 'She was heard singing from a distance.'
        \ex\label{middles5} \gll Hann brendist illa.\\
        he burn+st.\Pst{} badly\\
        \trans 'He was badly burnt.'\\
        \ex\label{middles6} \gll Bókin fæst ikki.\\
        book-\Def{} get+st.\Prs{} not\\
        \trans 'It is impossible to get the book.'\\
        \ex\label{middles7} \gll Dyrnar opnaðust knappliga.\\
        door-\Def.\Pl{} open+st.\Pl.\Pst{} suddenly\\
        \trans 'The doors suddenly opened.'\\
    \end{xlist}
\end{exe}

These last examples is what corresponds most closely to middles in other languages
including English. An important difference to note between middle forms like the ones
in example~\pref{middles} and regular passives is that the former are non-agentive while
the latter are agentive. The agent is normally understood (and can be expressed with
a prepositional phrase, but this is frequently not done) from a regular passive, but not from the middle.
It is even considered ungrammatical to express an agent in middle constructions. We'll
mention a couple of examples illustrating the contrast between a middle and a regular passive:
\begin{exe}
    \ex\label{middle_passive1}
    \begin{xlist}
        \ex \gll * Bókin seldist av H. N. Jacobsens Bókahandli.\\
        {} book-\Def{} sell+st.\Pst{} by H. N. Jacobsen's bookstore\\
        \ex \gll Bókin varð seld av H. N. Jacobsens Bókahandli.\\
        book-\Def{} was sell.\Pst{} by H. N. Jacobsen's bookstore\\ 
        \trans 'The book was sold by H. N. Jacobsen's bookstore.'\\
    \end{xlist}
    \ex\label{middle_passive2}
    \begin{xlist}
        \ex \gll * Jóhanna brendist av óvinum sínum.\\
        {} Johanna burn+st.\Pst{} by enemy.\Pl{} her.\Refl\\
        \ex \gll Jóhanna varð brend av óvinum sínum.\\
        Johanna was burn.\Pst.\Ptcp{} by enemy.\Pl{} her.\Refl\\ 
        \trans 'Johanna was burned by her enemies.'\\
    \end{xlist}
\end{exe}

Here we see that it is not possible to specify an agent in a middle form in the way we
can do with a regular passive construction.

Finally we can mention that it can be difficult to predict what type of meaning an
\emph{-st}-form of a given verb will have. For instance the same \emph{-st}-form can
have two different meanings:
\begin{exe}
    \ex\label{kennast}
    \begin{xlist}
        \item \gll Tey kennast væl.\\
        they know+st.\Prs{} well\\
        \trans 'They know each other well.'
        \item \gll Tey kennast á málinum.\\
        they know+st.\Prs{} on language-\Def\\
        \trans 'They can be recognised by the way they speak.'
    \end{xlist}
\end{exe}

\subsection{Role and Reference Grammar}

Before we continue on to the structure of the lexicon let's analyse the middle constructions in example~\pref{middles}
in terms of Role and Reference Grammar (RRG), because this will be the basis for the rules in our parser.

\subsubsection*{Example~\pref{middles1}: Íbúð ynskist til leigu}

This sentence only consists of a single clause. The main verb, the predicate, in the clause
is \emph{-st}-form \emph{ynskist}, which is based on the verb \emph{ynskja} 'wish'. 
In RRG terms we call this the nucleus (\Nuc) of the clause.
The nucleus has two arguments (\Arg), the noun \emph{íbúð} 'apartment' and the prepositional
phrase \emph{til leigu} 'for rent'. The nucleus together with these two arguments form the core (\Core)
of the clause. We use a tree to give a quick overview of the sentence structure:

\begin{exe}
    \ex
    \psset{treesep=2ex, levelsep=3em}
    \SENTENCE{
        \CLAUSE{
        \CORE{
            \ARG{\WORD(NP){Íbúð}}
            \NUC{ynskist}
            \ARG{\FanEnd{PP}{til leigu}}
        }
        }
}
\end{exe}

For the subsequent examples we won't write out the analysis like we have done above, but
just illustrate the structure with a diagram, i.e. an RRG tree.

\subsubsection*{Example~\pref{middles2}: Bókin seldist væl}

\begin{exe}
    \ex
    \psset{treesep=2ex, levelsep=3em}
    \SENTENCE{
        \CLAUSE{
        \CORE{
            \ARG{\WORD(NP){Bókin}}
            \NUC{seldist}
            \ARG{\WORD(AdvP){væl}}
        }
        }
}
\end{exe}

\subsubsection*{Example~\pref{middles3}: Her skal eitt hús byggjast}

\begin{exe}
    \ex
    \psset{treesep=2ex, levelsep=3em}
    \TOP{
        \lPERIPH{4}{PP}{\WORD{Her}}
    \SENTENCE{
        \CLAUSE{
        \CORE{
            \NUC{skal}
            \ARG{\WORD(NP){eitt hús}}
            \NUC{byggjast}
        }
        }
    }
    }
\dolinks
\end{exe}

\subsubsection*{Example~\pref{middles4}: Hon hoyrdist syngja langa leið}

\begin{exe}
    \ex
    \psset{treesep=2ex, levelsep=3em}
    \SENTENCE{
        \CLAUSE{
        \CORE{
            \ARG{\WORD(NP){Hon}}
            \NUC{hoyrdist}
        }
        \CORE{
            \NUC{syngja}
            \ARG{\FanEnd{AdjP}{langa leið}}
        }
        }
}
\end{exe}

\subsubsection*{Example~\pref{middles5}: Hann brendist illa}

\begin{exe}
    \ex
    \psset{treesep=2ex, levelsep=3em}
    \SENTENCE{
        \CLAUSE{
        \CORE{
            \ARG{\WORD(NP){Hann}}
            \NUC{brendist}
            \ARG{\WORD(AdvP){illa}}
        }
        }
}
\end{exe}

\subsubsection*{Example~\pref{middles6}: Bókin fæst ikki}

\begin{exe}
    \ex
    \psset{treesep=2ex, levelsep=3em}
    \SENTENCE{
        \CLAUSE{
        \CORE{
            \ARG{\WORD(NP){Bókin}}
            \NUC{fæst}
            \ARG{\WORD(AdvP){ikki}}
        }
        }
}
\end{exe}

\subsubsection*{Example~\pref{middles7}: Dyrnar opnaðust knappliga}

\begin{exe}
    \ex
    \psset{treesep=2ex, levelsep=3em}
    \SENTENCE{
        \CLAUSE{
        \CORE{
            \ARG{\WORD(NP){Dyrnar}}
            \NUC{opnaðust}
            \ARG{\WORD(AdvP){knappliga}}
        }
        }
}
\end{exe}




\section{Implementation of the Lexicon and Grammar}

In this section we will first specify a lexicon to be used with the parser,
and based on this lexicon we build up a feature based grammar that we can use
with the parsers in NLTK (Natural Language ToolKit).

\subsection{Lexicon}

For our parser we need to define a number of features for nouns, pronouns, verbs,
prepositions, adjectives, determiners and adverbs. The features we will record are as follows:
\begin{exe}
    \ex\label{features}
    \begin{xlist}
    \ex word; the word itself in the form it is found in the text
    \ex part of speech type; noun (n), verb (v), etc.
    \ex definiteness; \Def-{} / \Def+{}
    \ex "middleness"; \Mid-{} / \Mid+{}
    \ex person; \First{} (first), \Second{} (second), or \Third{} (third)
    \ex gender; \M{} (masculine), \F{} (feminine), or \N{} (neuter) 
    \ex number; \Sg{} (singular) or \Pl{} (plural)
    \ex case; \Nom{} (nominative), \Acc{} (accusative), \Dat{} (dative), or \Gen{} (genitive)
    \ex noun type; \Count{} (countable noun) or \Mass{} (mass noun)
    \ex verb type; \Fin-{} (non-finite) or \Fin+{} (finite)
    \ex tense; \Pst{} (past) or \Prs{} (present) 
    \ex logical structure
    \end{xlist}
\end{exe}

Obviously, not all of these features will be relevant for all types of words, and in that case
we will write \emph{NA} for that feature. In the lexicon we give here we'll include
the words appearing in the examples in~\pref{middles}, and before we get to the lexicon itself
we list the words in their base form:
\begin{table}[!htbp]
    \begin{tabular}{rl}
        {nouns} & íbúð, leiga, bók, hús, leið, dyr \\
        {determiners} & ein \\
        {pronouns} & hon, hann \\
        {verbs} & ynskja, selja, skula, byggja, hoyra, syngja, brenna, fáa, opna \\
        {prepositions} & til \\
        {adjectives} & langur \\
        {adverbs} & væl, illa, ikki, her, knappliga \\
    \end{tabular}
\end{table}

Now we will move on and give each word as it appears in the examples in~\pref{middles}
and define the features as stated above in Example~\pref{features}.

\subsubsection{Nouns}

\begin{table}
    \centering
    \caption{The lexical features of the noun \emph{íbúð}.}
    \begin{tabular}{rl}
        \toprule
        Word & íbúð\\
        \midrule
        POS type & n\\
        \Def & \Def-\\
        \Mid & NA\\
        Person & \Third\\
        Gender & \F{}\\
        Number & \Sg{}\\
        Case & \Nom/\Acc/\Dat{}\\
        Tense & NA\\
        Logical structure & NA\\
        \bottomrule
    \end{tabular}
\end{table}


\begin{table}
    \centering
    \caption{The lexical features of the noun \emph{leigu}.}
    \begin{tabular}{rl}
        \toprule
        Word & leigu\\
        \midrule
        POS type & n\\
        \Def & \Def-\\
        \Mid & NA\\
        Person & \Third\\
        Gender & \F{}\\
        Number & \Sg{}\\
        Case & \Acc/\Dat/\Gen{} \\
        Tense & NA\\
        Logical structure & NA\\
        \bottomrule
    \end{tabular}
\end{table}

\begin{table}
    \centering
    \caption{The lexical features of the noun \emph{bókin}.}
    \begin{tabular}{rl}
        \toprule
        Word & bókin\\
        \midrule
        POS type & n\\
        \Def & \Def+\\
        \Mid & NA\\
        Person & \Third\\
        Gender & \F{}\\
        Number & \Sg{}\\
        Case & \Nom{} \\
        Tense & NA\\
        Logical structure & NA\\
        \bottomrule
    \end{tabular}
\end{table}

\begin{table}
    \centering
    \caption{The lexical features of the noun \emph{hús}.}
    \begin{tabular}{rl}
        \toprule
        Word & hús\\
        \midrule
        POS type & n\\
        \Def & \Def-\\
        \Mid & NA\\
        Person & \Third\\
        Gender & \N{}\\
        Number & \Sg{}\\
        Case & \Nom/\Acc{} \\
        Tense & NA\\
        Logical structure & NA\\
        \bottomrule
    \end{tabular}
\end{table}

\begin{table}
    \centering
    \caption{The lexical features of the noun \emph{leið}.}
    \begin{tabular}{rl}
        \toprule
        Word & leið\\
        \midrule
        POS type & n\\
        \Def & \Def-\\
        \Mid & NA\\
        Person & \Third\\
        Gender & \F{}\\
        Number & \Sg{}\\
        Case & \Nom/\Acc/\Dat{} \\
        Tense & NA\\
        Logical structure & NA\\
        \bottomrule
    \end{tabular}
\end{table}

\begin{table}[!htbp]
    \centering
    \caption{The lexical features of the noun \emph{dyrnar}.}
    \begin{tabular}{rl}
        \toprule
        Word & dyrnar\\
        \midrule
        POS type & n\\
        \Def & \Def+\\
        \Mid & NA\\
        Person & \Third\\
        Gender & \F{}\\
        Number & \Pl{}\\
        Case & \Nom/\Acc{} \\
        Tense & NA\\
        Logical structure & NA\\
        \bottomrule
    \end{tabular}
\end{table}



\subsubsection{Determiners}

\begin{table}
    \centering
    \caption{The lexical features of the determiner \emph{eitt}.}
    \begin{tabular}{rl}
        \toprule
        Word & eitt\\
        \midrule
        POS type & \Det\\
        \Def & \Def-\\
        \Mid & NA\\
        Person & \Third\\
        Gender & \N{}\\
        Number & \Sg{}\\
        Case & \Nom/\Acc{} \\
        Tense & NA\\
        Logical structure & NA\\
        \toprule
    \end{tabular}
\end{table}


\subsubsection{Pronouns}

\begin{table}
    \centering
    \caption{The lexical features of the pronoun \emph{hon}.}
    \begin{tabular}{rl}
        \toprule
        Word & hon\\
        \midrule
        POS type & \Pro\\
        \Def & \Def{+/-}\\
        \Mid & NA\\
        Person & \Third\\
        Gender & \F{}\\
        Number & \Sg{}\\
        Case & \Nom \\
        Tense & NA\\
        Logical structure & NA\\
        \bottomrule
    \end{tabular}
\end{table}

\begin{table}
    \centering
    \caption{The lexical features of the pronoun \emph{hann}.}
    \begin{tabular}{rl}
        \toprule
        Word & hann\\
        \midrule
        POS type & \Pro\\
        \Def & \Def{+/-}\\
        \Mid & NA\\
        Person & \Third\\
        Gender & \M{}\\
        Number & \Sg{}\\
        Case & \Nom/\Acc \\
        Noun type & NA\\
        Verb type & NA\\
        Tense & NA\\
        Logical structure & NA\\
        \bottomrule
    \end{tabular}
\end{table}


\subsubsection{Verbs}

\begin{table}
    \centering
    \caption{The lexical features of the verb \emph{ynskist}.}
    \begin{tabular}{rl}
        \toprule
        Word & ynskist\\
        \midrule
        POS type & v\\
        \Def & NA\\
        \Mid & \Mid+\\
        Person & \Third\\
        Gender & NA\\
        Number & \Sg{}\\
        Case & NA \\
        Tense & \Prs{}\\
        Logical structure & \pred{wish}($\varnothing$,y)\\
    \end{tabular}
\end{table}

\begin{table}
    \centering
    \caption{The lexical features of the verb \emph{seldist}.}
    \begin{tabular}{rl}
        \toprule
        Word & seldist\\
        \midrule
        POS type & v\\
        \Def & NA\\
        \Mid & \Mid+\\
        Person & \Third\\
        Gender & NA\\
        Number & \Sg{}\\
        Case & NA \\
        Tense & \Pst{}\\
        Logical structure & \prddo($\varnothing$,[\pred{sell}(y)])\\
        \bottomrule
    \end{tabular}
\end{table}

\begin{table}
    \centering
    \caption{The lexical features of the verb \emph{skal}.}
    \begin{tabular}{rl}
        \toprule
        Word & skal\\
        \midrule
        POS type & \Aux{}\\
        \Def & NA\\
        \Mid & \Mid-\\
        Person & \Third\\
        Gender & NA\\
        Number & \Sg{}\\
        Case & NA \\
        Noun type & NA\\
        Verb type & \Fin+\\
        Tense & \Prs{}\\
        Logical structure & \pred{be}(x,y)\\
        \bottomrule
    \end{tabular}
\end{table}

\begin{table}
    \centering
    \caption{The lexical features of the verb \emph{byggjast}.}
    \begin{tabular}{rl}
        \toprule
        Word & byggjast\\
        \midrule
        POS type & v\\
        \Def & NA\\
        \Mid & \Mid+\\
        Person & \Third\\
        Gender & NA\\
        Number & \Sg{}\\
        Case & NA \\
        Noun type & NA\\
        Verb type & \Fin+\\
        Tense & \Prs{}\\
        Logical structure & \prddo($\varnothing$,[\pred{build}(y)])\\
        \bottomrule
    \end{tabular}
\end{table}

\begin{table}
    \centering
    \caption{The lexical features of the verb \emph{hoyrdist}.}
    \begin{tabular}{rl}
        \toprule
        Word & hoyrdist\\
        \midrule
        POS type & v\\
        \Def & NA\\
        \Mid & \Mid+\\
        Person & \Third\\
        Gender & NA\\
        Number & \Sg{}\\
        Case & NA \\
        Tense & \Pst{}\\
        Logical structure & \pred{hear}($\varnothing$,y)\\
        \bottomrule
    \end{tabular}
\end{table}

\begin{table}
    \centering
    \caption{The lexical features of the verb \emph{syngja}.}
    \begin{tabular}{rl}
        \toprule
        Word & syngja\\
        \midrule
        POS type & v\\
        \Def & NA\\
        \Mid & \Mid-\\
        Person & \Third\\
        Gender & NA\\
        Number & \Sg{}\\
        Case & NA \\
        Tense & \Prs{}\\
        Logical structure & \prddo(x,[\pred{sing}(x,(y))])\\
        \bottomrule
    \end{tabular}
\end{table}

\begin{table}
    \centering
    \caption{The lexical features of the verb \emph{brendist}.}
    \begin{tabular}{rl}
        \toprule
        Word & brendist\\
        \midrule
        POS type & v\\
        \Def & NA\\
        \Mid & \Mid+\\
        Person & \Third\\
        Gender & NA\\
        Number & \Sg{}\\
        Case & NA \\
        Tense & \Pst{}\\
        Logical structure & \prddo($\varnothing$,[\pred{burn}(y)])\\
        \bottomrule
    \end{tabular}
\end{table}

\begin{table}
    \centering
    \caption{The lexical features of the verb \emph{fæst}.}
    \begin{tabular}{rl}
        \toprule
        Word & fæst\\
        \midrule
        POS type & v\\
        \Def & NA\\
        \Mid & \Mid+\\
        Person & \Third\\
        Gender & NA\\
        Number & \Sg{}\\
        Case & NA \\
        Tense & \Prs{}\\
        Logical structure & \pred{get}(x,y)\\
        \bottomrule
    \end{tabular}
\end{table}

\begin{table}
    \centering
    \caption{The lexical features of the verb \emph{opnaðust}.}
    \begin{tabular}{rl}
        \toprule
        Word & opnaðust\\
        \midrule
        POS type & v\\
        \Def & NA\\
        \Mid & \Mid+\\
        Person & \Third\\
        Gender & NA\\
        Number & \Pl{}\\
        Case & NA \\
        Noun type & NA\\
        Verb type & \Fin+\\
        Tense & \Pst{}\\
        Logical structure & \prddo($\varnothing$,[\pred{open}(y)])\\
        \bottomrule
    \end{tabular}
\end{table}



\subsubsection{Prepositions}

\begin{table}
    \centering
    \caption{The lexical features of the preposition \emph{til}.}
    \begin{tabular}{rl}
        \toprule
        Word & til\\
        \midrule
        POS type & \Prep\\
        \Def & NA\\
        \Mid & NA\\
        Person & NA\\
        Gender & NA\\
        Number & NA\\
        Case & NA \\
        Tense & NA\\
        Logical structure & \pred{to}(x)\\
        \bottomrule
    \end{tabular}
\end{table}


\subsubsection{Adjectives}

\begin{table}
    \centering
    \caption{The lexical features of the adjective \emph{langa}.}
    \begin{tabular}{rl}
        \toprule
        Word & langa\\
        \midrule
        POS type & \Adj\\
        \Def & \Def-\\
        \Mid & NA\\
        Person & \Third\\
        Gender & \F{}\\
        Number & \Sg\\
        Case & \Acc \\
        Tense & NA\\
        Logical structure & NA\\
        \bottomrule
    \end{tabular}
\end{table}


\subsubsection{Adverbs}

\begin{table}
    \centering
    \caption{The lexical features of the adverb \emph{væl}.}
    \begin{tabular}{rl}
        \toprule
        Word & væl\\
        \midrule
        POS type & \Adv\\
        \Def & NA\\
        \Mid & NA\\
        Person & NA\\
        Gender & NA\\
        Number & NA\\
        Case & NA \\
        Tense & NA\\
        Logical structure & NA\\
        \bottomrule
    \end{tabular}
\end{table}

\begin{table}
    \centering
    \caption{The lexical features of the adverb \emph{illa}.}
    \begin{tabular}{rl}
        \toprule
        Word & illa\\
        \midrule
        POS type & \Adv\\
        \Def & NA\\
        \Mid & NA\\
        Person & NA\\
        Gender & NA\\
        Number & NA\\
        Case & NA \\
        Tense & NA\\
        Logical structure & NA\\
        \bottomrule
    \end{tabular}
\end{table}

\begin{table}
    \centering
    \caption{The lexical features of the adverb \emph{ikki}.}
    \begin{tabular}{rl}
        \toprule
        Word & ikki\\
        \midrule
        POS type & \Adv\\
        \Def & NA\\
        \Mid & NA\\
        Person & NA\\
        Gender & NA\\
        Number & NA\\
        Case & NA \\
        Tense & NA\\
        Logical structure & NA\\
        \bottomrule
    \end{tabular}
\end{table}

\begin{table}
    \centering
    \caption{The lexical features of the adverb \emph{her}.}
    \begin{tabular}{rl}
        \toprule
        Word & her\\
        \midrule
        POS type & \Adv\\
        \Def & NA\\
        \Mid & NA\\
        Person & NA\\
        Gender & NA\\
        Number & NA\\
        Case & NA \\
        Tense & NA\\
        Logical structure & NA\\
        \bottomrule
    \end{tabular}
\end{table}

\begin{table}
    \centering
    \caption{The lexical features of the adverb \emph{knappliga}.}
    \begin{tabular}{rl}
        \toprule
        Word & knappliga\\
        \midrule
        POS type & \Adv\\
        \Def & NA\\
        \Mid & NA\\
        Person & NA\\
        Gender & NA\\
        Number & NA\\
        Case & NA \\
        Tense & NA\\
        Logical structure & NA\\
        \bottomrule
    \end{tabular}
\end{table}


\subsection{Grammar}

We will now build up a feature based grammar for use with the parser. For this we will use
a so-called feature structure in NLTK, which basically is a Python dictionary with the
features as keys and the feature values as values, e.g.:
\begin{lstlisting}[caption={A feature structure showing lexical features for the word \emph{íbúð}.}]
    import nltk
    
    fs = nltk.FeatStruct(
        word='íbúð',
        pos='n',
        definiteness="def-",
        middle="na",
        person=3,
        gender='f',
        number='sg',
        case='nom',
        nounType='count',
        verbType='na',
        tense='na',
        ls='na')

    print(fs)

    [ case         = 'nom'   ]
    [ definiteness = 'def-'  ]
    [ gender       = 'f'     ]
    [ ls           = 'na'    ]
    [ middle       = 'na'    ]
    [ nounType     = 'count' ]
    [ number       = 'sg'    ]
    [ person       = 3       ]
    [ pos          = 'n'     ]
    [ tense        = 'na'    ]
    [ verbType     = 'na'    ]
    [ word         = 'íbúð'  ]
\end{lstlisting}


Faroese nouns agree with verbs in gender, person and number and with other nouns in
gender, person, number and case. Nouns agree with adjectives in case, number and gender.
To cover these agreement patterns we define an agreement feature like this:
\begin{lstlisting}[caption={A feature structure showing the word \emph{íbúð} with an \emph{agreement} feature.}]
        import nltk
        
        fs1 = nltk.FeatStruct(gender='f', person=3, number='sg')
        
        fs2 = nltk.FeatStruct(
            word='íbúð',
            pos='n',
            definiteness="def-",
            middle="na",
            agreement=fs1,
            case='nom',
            nounType='count',
            verbType='na',
            tense='na',
            ls='na')
    
        print(fs2)
    
        [                [ gender = 'f'  ] ]
        [ agreement    = [ number = 'sg' ] ]
        [                [ person = 3    ] ]
        [                                  ]
        [ case         = 'nom'             ]
        [ definiteness = 'def-'            ]
        [ ls           = 'na'              ]
        [ middle       = 'na'              ]
        [ nounType     = 'count'           ]
        [ pos          = 'n'               ]
        [ tense        = 'na'              ]
        [ verbType     = 'na'              ]
        [ word         = 'íbúð'            ]
\end{lstlisting}

Now we will move on to define the grammar itself in a file stating first the rules for
the middle constructions we have discussed in previous sections, then the lexicon in
terms of feature structures as discussed in this section. Below we show the finalised
grammar, complete will all grammar productions and lexical productions that cover the
middle constructions listed in Example~\pref{middles} on page~\pageref{middles}.

\lstinputlisting[caption={A feature based grammar for parsing 
                            of middle constructions in Faroese}]{code/faroese.fcfg}



\section{Implementation of the Parser}

The parser itself will utilise the chart parser that comes with NLTK as described in~\citep[Chapter 9]{nltk}.
Since we use the NLTK parser with the feature based grammar from the previous section,
the amount of code is minimal, hence we display it all here in Listing~\pref{parser}.

\lstinputlisting[caption={An RRG parser for middle constructions
                             in Faroese implemented in Python.},label={parser}]{code/parse.py}


\section{Testing of the Parser}




\section{Conclusion}

\subsection{Future Work}


\clearpage
\bibliographystyle{lin-v2/lin}    
\bibliography{references}

\end{document}